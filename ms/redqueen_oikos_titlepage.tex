\documentclass{article}\usepackage[]{graphicx}\usepackage[]{color}
\usepackage[top=1in,left=1in, right = 1in, footskip=1in]{geometry}

\title{Bridging the gap between theory and data: the Red Queen Hypothesis for sex}
\author{Sang Woo Park and Benjamin M. Bolker}

\usepackage{caption}
\captionsetup{%
   justification=raggedright,
   labelfont=bf,
  singlelinecheck=off
}

\usepackage{tabularx}

\usepackage{amsmath}
\usepackage{natbib}
\usepackage{hyperref}
\bibliographystyle{chicago}
\setcitestyle{aysep={}}
\setcitestyle{citesep={,}}
\date{}

\raggedright

\usepackage{bm}

\usepackage{afterpage}
\usepackage{pdflscape}

\newcommand{\etal}{\textit{et al.}}

\newcommand{\comment}[3]{\textcolor{#1}{\textbf{[#2: }\textit{#3}\textbf{]}}}
\newcommand{\bmb}[1]{\comment{cyan}{BMB}{#1}}
\newcommand{\swp}[1]{\comment{magenta}{SWP}{#1}}
\newcommand{\citetapos}[1]{\citeauthor{#1}'s \citeyearpar{#1}}

\newcommand{\fref}[1]{Fig.~\ref{fig:#1}}
\begin{document}

\maketitle

\section*{Abstract}

Sexual reproduction persists in nature despite its large cost.
The Red Queen Hypothesis postulates that parasite pressure maintains sexual reproduction in the host population by selecting for the ability to produce rare genotypes that are resistant to infection.
Mathematical models have been used to lay theoretical foundations for the hypothesis; empirical studies have confirmed these predictions.
For example, Lively used a simple host-parasite model to predict that the frequency of sexual hosts should be positively correlated with the prevalence of infection. 
Lively \textit{et al.} later confirmed the prediction through numerous field studies of snail-trematode systems in New Zealand.
In this study, we fit a simple metapopulation host-parasite coevolution model to three data sets, each representing a different snail-trematode system, by matching the observed prevalence of sexual reproduction and trematode infection among hosts.
Using the estimated parameters, we perform a power analysis to test the feasibility of observing the positive correlation predicted by Lively.
We discuss anomalies in the data that are poorly explained by the model and provide practical guidance to both modelers and empiricists.
Overall, our study suggests that a simple Red Queen model can only partially explain the observed relationships between parasite infection and the maintenance of sexual reproduction.

\section*{Keywords}

Red Queen Hypothesis, coevolution, host-parasite interaction, Approximate Bayesian Computation.

\section*{Declarations}

\subsection*{Acknowledgements}

We thank SHARCnet for providing computational resources. 

\subsection*{Funding}

This work is supported by The Natural Sciences and Engineering Research Council, Undergraduate Student Research Award (to SWP).

\end{document}
